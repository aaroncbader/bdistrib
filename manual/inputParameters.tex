\chapter{Input Parameters}
\label{ch:input}

\newcommand{\param}[5]{{\setlength{\parindent}{0cm} {\ttfamily \bfseries \hypertarget{#1}{#1}}\\{\it Type}: #2\\{\it Default}: #3\\{\it When it matters}: #4\\{\it Meaning}: #5}}
\newcommand{\myhrule}{{\setlength{\parindent}{0cm} \hrulefill }}

\newcommand{\true}{{\ttfamily .true.}}
\newcommand{\false}{{\ttfamily .false.}}

In this section we describe all the parameters which can be included in the input namelist. 

\myhrule

\section{Resolution parameters}

For any new set of surface geometries you consider, you should vary the resolution parameters in this section to make sure
they are large enough.  These parameters should be large enough that the code results you care about are unchanged under further
resolution increases.

\myhrule

\param{nu\_plasma}
{integer}
{64}
{Always}
{Number of grid points in poloidal angle used to evaluate the integral \todo{XXX} in the outer-to-plasma surface inductance matrix.
Often 64 is a good value.
It is resonable and common but not mandatory to use the same value for nu\_plasma, \parlink{nu\_middle}, and \parlink{nu\_outer}.}

\myhrule

\param{nu\_middle}
{integer}
{64}
{Always}
{Number of grid points in poloidal angle used to evaluate the integral \todo{XXX} in the outer-to-middle surface inductance matrix.
Often 64 is a good value.
It is resonable and common but not mandatory to use the same value for \parlink{nu\_plasma}, nu\_middle, and \parlink{nu\_outer}.}

\myhrule

\param{nu\_outer}
{integer}
{64}
{Always}
{Number of grid points in poloidal angle used to evaluate the integral \todo{XXX} in the outer-to-plasma surface and outer-to-middle surface inductance matrices.
Often 64 is a good value.
It is resonable and common but not mandatory to use the same value for \parlink{nu\_plasma}, \parlink{nu\_middle}, and nu\_outer.}

\myhrule

\param{nv\_plasma}
{integer}
{64}
{Always}
{Number of grid points in poloidal angle used to evaluate the integral \todo{XXX} in the outer-to-plasma surface inductance matrix.
A value of 64 is often good when the number of field periods is more than 1. A higher value like 256 may be needed when the number of field periods is 1.
It is resonable and common but not mandatory to use the same value for nv\_plasma, \parlink{nv\_middle}, and \parlink{nv\_outer}.}

\myhrule

\param{nv\_middle}
{integer}
{64}
{Always}
{Number of grid points in poloidal angle used to evaluate the integral \todo{XXX} in the outer-to-middle surface inductance matrix.
A value of 64 is often good when the number of field periods is more than 1. A higher value like 256 may be needed when the number of field periods is 1.
It is resonable and common but not mandatory to use the same value for \parlink{nv\_plasma}, nv\_middle, and \parlink{nv\_outer}.}

\myhrule

\param{nv\_outer}
{integer}
{64}
{Always}
{Number of grid points in poloidal angle used to evaluate the integral \todo{XXX} in the outer-to-plasma surface and outer-to-middle surface inductance matrices.
A value of 64 is often good when the number of field periods is more than 1. A higher value like 256 may be needed when the number of field periods is 1.
It is resonable and common but not mandatory to use the same value for \parlink{nv\_plasma}, \parlink{nv\_middle}, and nv\_outer.}

\myhrule

\param{mpol\_plasma}
{integer}
{8}
{Always}
{Maximum poloidal mode number to include for the Fourier basis functions on the plasma surface. The value can be positive or zero.
It is resonable and common but not mandatory to use the same value for {\ttfamily mpol\_plasma}, \parlink{mpol\_middle}, and \parlink{mpol\_outer}.}

\myhrule

\param{mpol\_middle}
{integer}
{8}
{Always}
{Maximum poloidal mode number to include for the Fourier basis functions on the middle surface. The value can be positive or zero.
It is resonable and common but not mandatory to use the same value for \parlink{mpol\_plasma}, {\ttfamily mpol\_middle}, and \parlink{mpol\_outer}.}

\myhrule

\param{mpol\_outer}
{integer}
{8}
{Always}
{Maximum poloidal mode number to include for the Fourier basis functions on the outer surface. The value can be positive or zero.
It is resonable and common but not mandatory to use the same value for \parlink{mpol\_plasma}, \parlink{mpol\_middle}, and {\ttfamily mpol\_outer}.}

\myhrule

\param{ntor\_plasma}
{integer}
{8}
{Always}
{Maximum toroidal mode number to include for the Fourier basis functions on the plasma surface. The value can be positive or zero.
It is resonable and common but not mandatory to use the same value for {\ttfamily ntor\_plasma}, \parlink{ntor\_middle}, and \parlink{ntor\_outer}.
You can set the three {\ttfamily ntor} parameters to zero to examine only axisymmetric modes.}

\myhrule

\param{ntor\_middle}
{integer}
{8}
{Always}
{Maximum toroidal mode number to include for the Fourier basis functions on the middle surface. The value can be positive or zero.
It is resonable and common but not mandatory to use the same value for \parlink{ntor\_plasma}, {\ttfamily ntor\_middle}, and \parlink{ntor\_outer}.
You can set the three {\ttfamily ntor} parameters to zero to examine only axisymmetric modes.}

\myhrule

\param{ntor\_outer}
{integer}
{8}
{Always}
{Maximum toroidal mode number to include for the Fourier basis functions on the outer surface. The value can be positive or zero.
It is resonable and common but not mandatory to use the same value for \parlink{ntor\_plasma}, \parlink{ntor\_middle}, and {\ttfamily ntor\_outer}.
You can set the three {\ttfamily ntor} parameters to zero to examine only axisymmetric modes.}

\section{Geometry parameters for the plasma surface}

\param{geometry\_option\_plasma}
{integer}
{0}
{Always}
{This option controls which type of geometry is used for the plasma surface.\\

{\ttfamily geometry\_option\_plasma} = 0: The plasma surface will be a plain circular torus. The major radius will be \parlink{R0\_plasma}.
     The minor radius will be \parlink{a\_plasma}.\\

{\ttfamily geometry\_option\_plasma} = 1: Identical to option 0. (This option exists just for consistency with \parlink{geometry\_option\_middle} and \parlink{geometry\_option\_outer}.)\\

{\ttfamily geometry\_option\_plasma} = 2: The plasma surface will be the last surface in the full radial grid of the \vmec~file specified by \parlink{wout\_ilename}.
The poloidal angle used will be the normal \vmec~angle which is not a straight-field-line coordinate.\\

{\ttfamily geometry\_option\_plasma} = 3: The plasma surface will be the last surface in the half radial grid of the \vmec~file specified by \parlink{wout\_filename}.
The poloidal angle used will be the normal \vmec~angle which is not a straight-field-line coordinate.\\

{\ttfamily geometry\_option\_plasma} = 4: The plasma surface will be the last surface in the half radial grid of the \vmec~file specified by \parlink{wout\_filename}.
The poloidal angle used will be the straight-field-line coordinate, obtained by shifting the normal \vmec~poloidal angle by \vmec's $\lambda$ quantity.\\

{\ttfamily geometry\_option\_plasma} = 5: The plasma surface will be the flux surface with normalized poloidal flux
\parlink{efit\_psiN} taken from the {\ttfamily efit} file specified by \parlink{efit\_filename}.
}

\myhrule

\param{R0\_plasma}
{real}
{10.0}
{Only when \parlink{geometry\_option\_plasma} is 0 or 1.}
{Major radius of the plasma surface, when this surface is a plain circular torus.}

\myhrule

\param{a\_plasma}
{real}
{0.5}
{Only when \parlink{geometry\_option\_plasma} is 0 or 1.}
{Minor radius of the plasma surface, when this surface is a plain circular torus.}

\myhrule

\param{nfp\_imposed}
{integer}
{1}
{Only when \parlink{geometry\_option\_plasma} is 0 or 1.}
{When the plasma surface is a plain circular torus, only toroidal mode numbers that are a multiple of this parameter will be considered.
This parameter thus plays a role like \vmec's {\ttfamily nfp} (number of field periods),
and is used when {\ttfamily nfp} is not already loaded from a \vmec~file.}

\myhrule

\param{wout\_filename}
{string}
{`'}
{Only when \parlink{geometry\_option\_plasma} is 2, 3, or 4.}
{Name of the \vmec~{\ttfamily wout} output file which will be used for the plasma surface.
You can use either a \netCDF~or {\ttfamily ASCII} format file.}

\myhrule

\param{efit\_filename}
{string}
{`'}
{Only when \parlink{geometry\_option\_plasma} is 5.}
{Name of the {\ttfamily efit} output file which will be used for the plasma surface.}

\myhrule

\param{efit\_psiN}
{real}
{0.98}
{Only when \parlink{geometry\_option\_plasma} is 5.}
{Value of normalized poloidal flux at which to select a flux surface from the {\ttfamily efit} input file.
A value of 1 corresponds to the last closed flux surface, and 0 corresponds to the magnetic axis.}

\myhrule

\param{efit\_num\_modes}
{integer}
{10}
{Only when \parlink{geometry\_option\_plasma} is 5.}
{Controls the number of Fourier modes used to represent $R(\theta)$ and $Z(\theta)$ for the shape of
the plasma surface. Each of these functions will be expanded in a sum of functions $\sin(m\theta)$ and $\cos(m\theta)$,
where $m$ ranges from 0 to {\ttfamily efit\_num\_modes}$-1$.}

\section{Geometry parameters for the middle surface}

\param{geometry\_option\_middle}
{integer}
{0}
{Always}
{This option controls which type of geometry is used for the middle surface.\\

{\ttfamily geometry\_option\_middle} = 0: The middle surface will be a plain circular torus. The major radius will be the 
same as the plasma surface: either \parlink{R0\_plasma} if \parlink{geometry\_option\_plasma} is 0 or 1, or {\ttfamily Rmajor\_p} from the \vmec~{\ttfamily wout} file
if  \parlink{geometry\_option\_plasma} is 2.
     The minor radius will be \parlink{a\_middle}.\\

{\ttfamily geometry\_option\_middle} = 1: Identical to option 0, except the major radius of the middle surface will be set by \parlink{R0\_middle}.\\

{\ttfamily geometry\_option\_middle} = 2: The middle surface will computing by expanding the plasma surface uniformly by a distance \parlink{separation\_middle}.\\

{\ttfamily geometry\_option\_middle} = 3: The middle surface will be the `coil' surface in the \nescoil~`nescin' input file specified by \parlink{nescin\_filename\_middle}.
}

\myhrule

\param{R0\_middle}
{real}
{10.0}
{Only when \parlink{geometry\_option\_middle} is 1.}
{Major radius of the middle surface, when this surface is a plain circular torus.}

\myhrule

\param{a\_middle}
{real}
{1.0}
{Only when \parlink{geometry\_option\_middle} is 0 or 1.}
{Minor radius of the middle surface, when this surface is a plain circular torus.}


\myhrule

\param{separation\_middle}
{real}
{0.2}
{Only when \parlink{geometry\_option\_middle} is 2.}
{Amount by which the middle surface is offset from the plasma surface.}

\myhrule

\param{nescin\_filename\_middle}
{string}
{`'}
{Only when \parlink{geometry\_option\_middle} is 3.}
{Name of a \nescoil~{\ttfamily nescin} input file. The coil surface from
this file will be used as the middle surface for \bdistrib.}


\section{Geometry parameters for the outer surface}

\param{geometry\_option\_outer}
{integer}
{0}
{Always}
{This option controls which type of geometry is used for the outer surface.\\

{\ttfamily geometry\_option\_outer} = 0: The outer surface will be a plain circular torus. The major radius will be \parlink{R0\_outer}.
     The minor radius will be \parlink{a\_outer}.\\

{\ttfamily geometry\_option\_outer} = 1: Identical to option 0.\\

{\ttfamily geometry\_option\_outer} = 2: The outer surface will computing by expanding the plasma surface uniformly by a distance \parlink{separation\_outer}.\\

{\ttfamily geometry\_option\_outer} = 3: The outer surface will be the `coil' surface from the \nescoil~`nescin' input file specified by \parlink{nescin\_filename\_outer}.
}

\myhrule

\param{R0\_outer}
{real}
{10.0}
{Only when \parlink{geometry\_option\_outer} is 1.}
{Major radius of the outer surface, when this surface is a plain circular torus.}

\myhrule

\param{a\_outer}
{real}
{1.5}
{Only when \parlink{geometry\_option\_outer} is 0 or 1.}
{Minor radius of the outer surface, when this surface is a plain circular torus.}

\myhrule

\param{separation\_outer}
{real}
{0.4}
{Only when \parlink{geometry\_option\_outer} is 2.}
{Amount by which the outer surface is offset from the plasma surface.  (Not the offset between the outer and \emph{middle} surfaces!).}

\myhrule

\param{nescin\_filename\_outer}
{string}
{`'}
{Only when \parlink{geometry\_option\_outer} is 3.}
{Name of a \nescoil~{\ttfamily nescin} input file. The coil surface from
this file will be used as the outer surface for \bdistrib.}

\section{Other parameters}

\myhrule

\param{basis\_set\_option}
{integer}
{1}
{Always}
{Determines which set of basis functions is used.\\

{\ttfamily basis\_set\_option} = 1: Use $\sin(2 \pi [mu+nv])$ basis functions.\\

{\ttfamily basis\_set\_option} = 2: Use $\cos(2 \pi [mu+nv])$ basis functions.\\

{\ttfamily basis\_set\_option} = 3: Use both $\sin(2 \pi [mu+nv])$ and $\cos(2 \pi [mu+nv])$ basis functions.
}

\myhrule

\param{pseudoinverse\_thresholds}
{real array}
{1e-12}
{Always}
{To form the pseudoinverse of the outer-to-middle inductance matrix, singular values larger than this threshold
will be replaced by their inverse, whereas singular values smaller than this threshold will be replaced by zero.
You can supply more than one threshold value in order to compare results for different choices.
Good threshold values are in the range 1e-6 to 1e-13, i.e. small compared to 1 but larger than machine precision.}

\myhrule

\param{n\_singular\_vectors\_to\_save}
{integer}
{12}
{Always}
{Number of columns of $U$ and $V$ to save in the output file, where $U$ and $V$ are the matrices of left and right singular vectors of the transfer matrix,
$T = U\Sigma V^{T}$.
Regardless of this parameter, all singular \emph{values} will be saved in the output file.}

\myhrule

\param{save\_level}
{integer}
{2}
{Always}
{Option related determining how many variables are saved in the \netCDF~output file.  The larger the value, the smaller the output file.\\

{\ttfamily save\_level} = 0: Save everything.\\

{\ttfamily save\_level} = 1: Save everything except the inductance matrices.\\

{\ttfamily save\_level} = 2: Save everything except the inductance matrices and the {\ttfamily drdu}, {\ttfamily drdv}, and {\ttfamily normal} arrays.
}

